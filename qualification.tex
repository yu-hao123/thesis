\documentclass[]{politex}
% ========== Opções ==========
% pnumromarab - Numeração de páginas usando algarismos romanos na parte pré-textual e arábicos na parte textual
% abnttoc - Forçar paginação no sumário conforme ABNT (inclui "p." na frente das páginas)
% normalnum - Numeração contínua de figuras e tabelas
%	(caso contrário, a numeração é reiniciada a cada capítulo)
% draftprint - Ajusta as margens para impressão de rascunhos
%	(reduz a margem interna)
% twosideprint - Ajusta as margens para impressão frente e verso
% capsec - Forçar letras maiúsculas no título das seções
% espacosimples - Documento usando espaçamento simples
% espacoduplo - Documento usando espaçamento duplo
%	(o padrão é usar espaçamento 1.5)
% times - Tenta usar a fonte Times New Roman para o corpo do texto
% noindentfirst - Não indenta o primeiro parágrafo dos capítulos/seções

% ========== Packages ==========
\usepackage[utf8]{inputenc}
\usepackage{amsmath,amsthm,amsfonts,amssymb}
\usepackage{graphicx,cite,enumerate,longtable}
\usepackage{libertinus}
\usepackage{booktabs}

\usepackage[T1]{fontenc}
\usepackage{textcomp}
\usepackage{lipsum} % lorem ipsum
\usepackage{xcolor}  % must come before soul
\usepackage{soul} % highlight text

% ========== Language options ==========
%\usepackage[brazil]{babel}
\usepackage[english]{babel}

% Package for including code in the document
\usepackage{listings}

% bookmarks in custom
\usepackage[
  colorlinks = true,
  linkcolor  = black,
  citecolor  = black,
  urlcolor   = magenta,
  linktoc    = all,
  bookmarks  = true,
  bookmarksnumbered = true,
  bookmarksopen = true
]{hyperref}

% reformats list of figures
\makeatletter
\renewcommand*\l@figure[2]{%
\@dottedtocline{1}{1.5em}{2.3em}{FIGURE~#1}{#2}%
}
\makeatother

% PRETEXTUAL AND NUMBERED BOOKMARKS FIX
\makeatletter
% --- PART 1: Fix for Numbered Chapters (after \sumario) ---
%
% The class's \ABNTaddcontentsline command creates broken links.
% We replace it with the standard, working \addcontentsline.
%
\renewcommand{\ABNTaddcontentsline}[3]{%
	\ifthenelse{\boolean{ABNTNextOutOfTOC}}%
		{\setboolean{ABNTNextOutOfTOC}{false}}%
		{%
		  \addcontentsline{#1}{#2}{#3}%
		}%
}

% --- PART 2: Fix for Pretextual Chapters (before \sumario) ---
%
% We patch \@schapter to manually create bookmarks, since aren't in the TOC.
%
\let\politex@original@schapter\@schapter
\def\@schapter#1{%
	\def\temptitle{#1}%
	\def\contentstitle{\contentsname}%

	\ifx\temptitle\contentstitle
		% This is the Table of Contents ("Sumário").
		% Run the original command *without* \pdfbookmark
		% to allow the TOC to build correctly.
		\politex@original@schapter{#1}%
	\else
		% This is "Resumo", "Abstract", "Lista de Figuras", etc.
		% manually create the PDF bookmark and its anchor.
		\pdfbookmark[0]{#1}{#1}%

		% Run the original command to print the title.
		% (Since ABNTaftertoc is false, this won't go to the TOC)
		\politex@original@schapter{#1}%
	\fi
}
\makeatother

% ========== ABNT (requer ABNTeX 2) ==========
%	http://www.ctan.org/tex-archive/macros/latex/contrib/abntex2
\usepackage[alf]{abntex2cite}

% Forçar o abntex2 a usar [ ] nas referências ao invés de ( )
%\citebrackets{[}{]}

% ========== Opções do documento ==========
% Título
\titulo{Identifiability of Respiratory Mechanics: Limitations of Single Cycle Estimation and Proposal of Multiple Cycle Maneuvers}

% Autor
\autor{Yu Hao Wang Xia}

% Para múltiplos autores (TCC)
%\autor{Nome Sobrenome\\%
%		Nome Sobrenome\\%
%		Nome Sobrenome}

% Orientador / Coorientador
\orientador{Prof.\ Dr.\ Marcelo Britto Passos Amato}
%\coorientador{Nome do coorientador (opcional)}

% Tipo de documento
%\tcc{Eletricista com ênfase em Sistemas Eletrônicos}
%\dissertacao{Engenharia Elétrica}
\teseDOC{Science}
%\teseLD
%\memorialLD

% Departamento e área de concentração
\departamento{None}
\areaConcentracao{Respiratory Sciences}

% Local
\local{São Paulo}

% Ano
\data{2025}

\begin{document}
% ========== Capa e folhas de rosto ==========
\capa%
%\falsafolhaderosto
\folhaderosto%

% ========== Folha de assinaturas (opcional) ==========
%\begin{folhadeaprovacao}
%	\assinatura{Prof.\ X}
%	\assinatura{Prof.\ Y}
%	\assinatura{Prof.\ Z}
%\end{folhadeaprovacao}

% ========== Ficha catalográfica ==========
% Fazer solicitação no site:
%	http://www.poli.usp.br/en/bibliotecas/servicos/catalogacao-na-publicacao.html

% ========== Dedicatória (opcional) ==========
%\dedicatoria{Dedicatória}

% ========== Agradecimentos ==========
%\begin{agradecimentos}
%Thanks
%\end{agradecimentos}

% ========== Epígrafe (opcional) ==========
\epigrafe{%
	\emph{``O Brasil é grande demais para sonharmos pequeno''}
	\begin{flushright}
		-{}- Ozires Silva
	\end{flushright}
}

% ========== Resumo ==========
\begin{resumo}
Resumo
%
\\[3\baselineskip]
%
\textbf{Palavras-Chave:} Palavra, Palavra, Palavra, Palavra, Palavra.
\end{resumo}

% ========== Abstract ==========
\begin{abstract}
Abstract
%
\\[3\baselineskip]
%
\textbf{Keywords:} Word, Word, Word, Word, Word.
\end{abstract}

% ========== Listas (opcional) ==========
\listadefiguras%
\listadetabelas%

% ========== Listas definidas pelo usuário (opcional) ==========
\begin{pretextualsection}{List of Symbols}
\begin{longtable}{ll}

AIC & Akaike information criterion \\
ARDS & acute respiratory distress syndrome \\
ARDSNet & acute respiratory distress syndrome network \\
ARX & autoregressive with external input \\
ARMAX & autoregressive moving average with external input \\
COVID-19 & coronavirus disease 2019 \\
CT & computerized tomography \\
FiO$_2$ & fraction of inspired oxygen \\
FIR & finite impulse response \\
FRC & functional residual capacity \\
IBW & ideal body weight \\
ICU & intensive care unit \\
LTI & linear and time-invariant \\
MSE & mean squared error \\
%NARX & nonlinear autoregressive with external input \\
NRMSE & normalized root mean squared error\\
OE & output error \\
PaO$_2$ & arterial oxygen partial pressure \\
PEEP & positive end-expiratory pressure\\
PCV & pressure controlled ventilation \\
PSV & pressure support ventilation \\
RR & respiratory rate \\
SpO$_2$ & peripheral oxygen saturation \\
VCV & volume controlled ventilation \\
WOB & work of breathing \\
\end{longtable}

\end{pretextualsection}

\begin{pretextualsection}{LIST OF SYMBOLS}
\onehalfspacing
\begin{longtable}{ll}
$P_{\text{total}}$ & respiratory system total pressure \\
$P_{\text{aw}}$ & airway pressure \\
$P_{\text{mus}}$ & respiratory muscles pressure \\
$P_0$ & positive end-expiratory pressure \\
$P_{\text{es}}$ & esophageal pressure \\
$P_{\text{pl}}$ & pleural pressure \\
$P_{\text{L}}$ & transpulmonary pressure \\
$R$ & respiratory system resistance \\
$E$ & respiratory system elastance \\
$C$ & respiratory system compliance \\
$E_{\text{L}}$ & lung elastance \\
$C_{\text{L}}$ & lung compliance \\
$E_{\text{cw}}$ & chest wall elastance \\
$C_{\text{cw}}$ & chest wall compliance \\
$V_{\text{T}}$ & tidal volume \\
$V$ & air volume \\
$\dot{V}$ & airflow \\
\end{longtable}
\end{pretextualsection}


% ========== Sumário ==========
\sumario%
% ========== Elementos textuais ==========

\chapter{INTRODUCTION}

Respiratory effort monitoring is recognized as an essential tool for individualizing
mechanical ventilator settings in lung and diaphragm protective ventilation strategies
~\cite{vanoosten2024, cornejo2024}.
Monitoring is crucial because both excessive and low inspiratory efforts are associated with
multiple types of lung and diaphragm injuries including patient self-inflicted lung injury (P-SILI)
and myotrauma~\cite{tonelli2025,yoshida2020b}, and mismatches/asynchronies can induce traumas to
both organs. Thus, the primary aim of such monitoring is to maintain in the patient a
moderate amount of respiratory drive to prevent mechanical ventilation-induced complications and
help in weaning~\cite{jonkman2020}.

Explore monitoring (esophageal~\cite{akoumianaki2014,mauri2016}, maneuver surrogates PO.1 and Pooc) and estimation methods.

There are multiple estimators that apply some form of
constrained optimization approach~\cite{vicario2016,victorjr2023,lv2025}.

Recent developments published clinical validations of such methods~\cite{lv2025}.

Analyze weaknesses and possible knowledge gaps in noninvasive estimation.

Show identifiability methods~\cite{raue2009,wieland2021}.

Thus the purpose of this work is to reproduce literature noninvasive muscle estimation,
explore the robustness and identifiability of such formulations and propose improvements.

\begin{figure}[ht]
	\caption{
        Space delimitation
    }\label{fig:circulo}
	\begin{center}
		\setlength{\unitlength}{9cm}
		\begin{picture}(1,1)
		\put(0,0){\line(0,1){1}}
		\put(0,0){\line(1,0){1}}
		\put(0,0){\line(1,1){1}}
		\put(0,0){\line(1,2){.5}}
		\put(0,0){\line(1,3){.3333}}
		\put(0,0){\line(1,4){.25}}
		\put(0,0){\line(1,5){.2}}
		\put(0,0){\line(1,6){.1667}}
		\put(0,0){\line(2,1){1}}
		\put(0,0){\line(2,3){.6667}}
		\put(0,0){\line(2,5){.4}}
		\put(0,0){\line(3,1){1}}
		\put(0,0){\line(3,2){1}}
		\put(0,0){\line(3,4){.75}}
		\put(0,0){\line(3,5){.6}}
		\put(0,0){\line(4,1){1}}
		\put(0,0){\line(4,3){1}}
		\put(0,0){\line(4,5){.8}}
		\put(0,0){\line(5,1){1}}
		\put(0,0){\line(5,2){1}}
		\put(0,0){\line(5,3){1}}
		\put(0,0){\line(5,4){1}}
		\put(0,0){\line(5,6){.8333}}
		\put(0,0){\line(6,1){1}}
		\put(0,0){\line(6,5){1}}
		\end{picture}
	\end{center}
    \vspace{2mm}
    {\small Source: Solid line - simulated data; dashed line - experimental data.}
\end{figure}

\chapter{HYPOTHESIS AND OBJECTIVES}

We hypothesize that:

\begin{itemize}
    \item Overly restrictive formulations in constrained optimization may cause
    overfitting and generate respiratory effort estimatives that deviate from the true physiological condition.
    \item The monotonic formulation proposed by~\cite{vicario2016} applied in isolated/single
    cycles is insufficient to guarantee a unique and physiologically consistent estimative.
    \item Incorporating multiple respiratory cycles or controlled maneuvers into the estimation algorithm reduces ambiguity.
    \item Identifiability analysis provides a complementary approach to evaluate the estimation algorithms,
    offering perspectives that cannot be obtained solely from statistical validation on experimental and clinical data.
\end{itemize}

The main objectives of this work are:

\begin{itemize}
    \item Reproduce and compare existing model-based constrained optimization formulations for
    noninvasive estimation of respiratory effort, comparing them using both experimental and clinical data through statistical analysis.
    \item Evaluate ambiguities and the identifiability of these respiratory effort estimation models and formulations.
    \item Propose methodological improvements to improve reliability and reduce ambiguity of noninvasive effort monitoring.
\end{itemize}
\chapter{MATERIALS AND METHODS}


\section{Algorithms formulation}

Assuming a first-order model for the respiratory system~\cite{bates2009}:

\begin{equation}
P_{\text{total}}(t) = P_{\text{aw}}(t) + P_{\text{mus}}(t) = E_{\text{rs}} \cdot V(t) + R_{\text{rs}} \cdot \dot{V}(t) + P_0
\label{eq:equation_motion}
\end{equation}


The explored \hl{(determininistic/modelling (?))} algorithms apply contrained optimization techniques
in the respiratory equation of motion.
The problem of estimating $P_{\text{mus}}$, $R$, $E$ within a single cycle is indetermined ($N+2$ variables for $N$ equations as will be detailed below),
but the explored algorithms argument is that by introducing constraints on the $P_{\text{mus}}$ waveform format and the ranges of possible resistances and compliances, the problem becomes solvable.

~\cite{vicario2016} formulated the estimation problem as a constrained optimization problem with cost function:
\begin{equation}
J = \sum_{k=1}^{N}
\left(
P_{\text{aw}}(t_k) -\left(R\dot{V}(t_k) + E V(t_k) + \tilde{P}_{\text{mus}}(t_k)\right)
\right)^2
\label{eq:cost_function}
\end{equation}
to be minimized subject to the following constraints:
\begin{subequations}
\begin{align}
\tilde{P}_{\text{mus}}(t_{k+1}) - \tilde{P}_{\text{mus}}(t_k) &\le 0,
&& \text{for } k = 1,2,\ldots,m-1 \label{eq:constraint1} \\
\tilde{P}_{\text{mus}}(t_{k+1}) - \tilde{P}_{\text{mus}}(t_k) &\ge 0,
&& \text{for } k = m,m+1,\ldots,q-1 \label{eq:constraint2} \\
\tilde{P}_{\text{mus}}(t_{k+1}) - \tilde{P}_{\text{mus}}(t_k) &= 0,
&& \text{for } k = q,q+1,\ldots,N-1 \label{eq:constraint3}
\end{align}
\end{subequations}
where $t_k$ denotes the $k$th time sample, $N$ is the total number of time samples in the breath.
\hl{The motivation behind these constraints is to define monotonicity of the muscle pressure format?} (Equations~\ref{eq:constraint1}–\ref{eq:constraint3}) is the
\hl{The parameters $t_m$ and $t_q$ define:}

The unknown variables over which $J$ is minimized are $R, E, \tilde{P}_{\text{mus}}(t_1), \ldots, \tilde{P}_{\text{mus}}(t_N)$.
Additional constraints limit $\tilde{P}_{\text{mus}}(t_k)$, $R$ and $E$ to positive values and within their physiological ranges:
\begin{subequations}
\begin{align}
R_{\text{min}} &\le R \le R_{\text{max}} \label{eq:R_bounds} \\
E_{\text{min}} &\le E \le E_{\text{max}} \label{eq:E_bounds} \\
\tilde{P}_{\text{min}} &\le \tilde{P}_{\text{mus}}(t_k) \le \tilde{P}_{\text{max}}. \label{eq:pmus_bounds}
\end{align}
\end{subequations}

LS optimization problem. In other words, Another way to interpret the Equation~\ref{eq:cost_function} is that
the optimization solver is trying to obtain the closest possible estimative of $P_{\text{aw}}$
given the equation of motion, flow, volume and constraints on the unknown variables.
We can call the second term of the Equation~\ref{eq:cost_function} the estimative of airway pressure:





\begin{equation}
P_{\text{aw}}^{\text{est}}(t_k)  = \left(R\dot{V}(t_k) + E V(t_k) + \tilde{P}_{\text{mus}}(t_k)\right)
\label{eq:paw_est}
\end{equation}

~\cite{victorjr2023} improved the estimation and computational time by adding switching times:
explain Pawest -> equation


Another explored algorithm is the one proposed by~\cite{lv2025}. It is given as follows:

\section{Implementation environment}

Gurobi, YALMIP, MATLAB softwares, hardware used.

Code repository available in github/pmus-benchmarking.git (instructions of use, reproducibility).

\section{Proposed improvements}

This work proposes a improvement of using the information of more than one cycle,
as opposed to the current standard/understandings on non-invasive respiratory effort estimation.
For that purpose, excitation maneuvers will be applied in the experimental setting
with the objective to create variability and \hl{add more information} to the optimization problem.

\subsection{Maneuvers}

Description of all attempted maneuvers: P0.1, inspiratory pause, PS increase.

\subsection{Dual cycle estimation}

\section{Data acquisition}
\subsection{Benchtop experiment}
\begin{itemize}
    \item ASL 5000\textsuperscript{\texttrademark} Breathing Simulator is connected to a mechanical ventilator FlexiMag Max 700
(Magnamed\textsuperscript{\textregistered}, Sao Paulo, Brazil) configuration steps (pressure, resistance, compliance), ventilation modes, physical setup.
    \item Sampling frequency, acquisition duration, signal pre-processing.
\end{itemize}

\subsection{Patient data analysis (Restrospective (?))}
\begin{itemize}
    \item Dataset origin, ethical comittee number, anonymization.
    \item Patients description, inclusion/exclusion criteria, condition, modes, duration.
\end{itemize}

\section{Evaluation methods}

\subsection{Profile likelihood}

\subsection{Cost function heatmap}

\subsection{PMUS variability within cost valley}

\subsection{Global statistics}

PMUS Amplitudes and PTPmus Bland-Altman plot in experimental and patient data. \\
Performance comparison of the three algorithms Victor, Lv, Xia.

\chapter{RESULTS}


\begin{figure}[ht]
    \caption{
        MIQP-based estimation for a single respiratory cycle.
    }\label{fig:C4_miqp_best_cycle}
    \begin{center}
        \includegraphics[width=.8\textwidth]{chapters/figures/C4_miqp_best_cycle.png}
    \end{center}
    %{\small Source: Produced by the author (2025)}
\end{figure}{}


\begin{figure}[ht]
    \caption{
        MIQP-based estimation for a single respiratory cycle with fixed $R$ and $C$ set to their true values (obtained from LS using ASL ${P}_{mus}$).
    }\label{fig:C4_miqp_true_cycle}
    \begin{center}
        \includegraphics[width=.75\textwidth]{chapters/figures/C4_miqp_true_cycle.png}
    \end{center}
    %{\small Source: Produced by the author (2025)}
\end{figure}{}

\lipsum[1]

\begin{figure}[ht]
    \caption{
        Heatmap of the MIQP estimator cost function for a single respiratory cycle.
    }\label{fig:C4_miqp_heatmap_cycle}
    \begin{center}
        \includegraphics[width=.75\textwidth]{chapters/figures/C4_heatmap_miqp_cycle.png}
    \end{center}
    %{\small Source: Produced by the author (2025)}
\end{figure}{}
\begin{figure}[ht]
    \caption{
        Heatmap of the N-PMUS estimator cost function for a single respiratory cycle.
    }\label{fig:C4_cubic_heatmap_cycle}
    \begin{center}
        \includegraphics[width=.75\textwidth]{chapters/figures/C4_heatmap_cubic_cycle.png}
    \end{center}
    %{\small Source: Produced by the author (2025)}
\end{figure}{}

% ========== Referências ==========
% --- IEEE ---
%	http://www.ctan.org/tex-archive/macros/latex/contrib/IEEEtran
%\bibliographystyle{IEEEbib}

% --- ABNT (requer ABNTeX 2) ---
%	http://www.ctan.org/tex-archive/macros/latex/contrib/abntex2
\bibliographystyle{abntex2-num}
\bibliography{bibliography_thesis}

% ========== Apêndices (opcional) ==========
\apendice%
\chapter{Alpha}

% ========== Anexos (opcional) ==========
\anexo%
\chapter{Alpha}

\end{document}
