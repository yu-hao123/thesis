\chapter{MATERIALS AND METHODS}


\section{Algorithms formulation}

Assuming a first-order model for the respiratory system~\cite{bates2009}:

\begin{equation}
P_{\text{total}}(t) = P_{\text{aw}}(t) + P_{\text{mus}}(t) = E_{\text{rs}} \cdot V(t) + R_{\text{rs}} \cdot \dot{V}(t) + P_0
\label{eq:equation_motion}
\end{equation}


The explored \hl{(determininistic/modelling (?))} algorithms apply contrained optimization techniques
in the respiratory equation of motion.
The problem of estimating $P_{\text{mus}}$, $R$, $E$ within a single cycle is indetermined ($N+2$ variables for $N$ equations as will be detailed below),
but the explored algorithms argument is that by introducing constraints on the $P_{\text{mus}}$ waveform format and the ranges of possible resistances and compliances, the problem becomes solvable.

~\cite{vicario2016} formulated the estimation problem as a constrained optimization problem with cost function:
\begin{equation}
J = \sum_{k=1}^{N}
\left(
P_{\text{aw}}(t_k) -\left(R\dot{V}(t_k) + E V(t_k) + \tilde{P}_{\text{mus}}(t_k)\right)
\right)^2
\label{eq:cost_function}
\end{equation}
to be minimized subject to the following constraints:
\begin{subequations}
\begin{align}
\tilde{P}_{\text{mus}}(t_{k+1}) - \tilde{P}_{\text{mus}}(t_k) &\le 0,
&& \text{for } k = 1,2,\ldots,m-1 \label{eq:constraint1} \\
\tilde{P}_{\text{mus}}(t_{k+1}) - \tilde{P}_{\text{mus}}(t_k) &\ge 0,
&& \text{for } k = m,m+1,\ldots,q-1 \label{eq:constraint2} \\
\tilde{P}_{\text{mus}}(t_{k+1}) - \tilde{P}_{\text{mus}}(t_k) &= 0,
&& \text{for } k = q,q+1,\ldots,N-1 \label{eq:constraint3}
\end{align}
\end{subequations}
where $t_k$ denotes the $k$th time sample, $N$ is the total number of time samples in the breath.
\hl{The motivation behind these constraints is to define monotonicity of the muscle pressure format?} (Equations~\ref{eq:constraint1}–\ref{eq:constraint3}) is the
\hl{The parameters $t_m$ and $t_q$ define:}

The unknown variables over which $J$ is minimized are $R, E, \tilde{P}_{\text{mus}}(t_1), \ldots, \tilde{P}_{\text{mus}}(t_N)$.
Additional constraints limit $\tilde{P}_{\text{mus}}(t_k)$, $R$ and $E$ to positive values and within their physiological ranges:
\begin{subequations}
\begin{align}
R_{\text{min}} &\le R \le R_{\text{max}} \label{eq:R_bounds} \\
E_{\text{min}} &\le E \le E_{\text{max}} \label{eq:E_bounds} \\
\tilde{P}_{\text{min}} &\le \tilde{P}_{\text{mus}}(t_k) \le \tilde{P}_{\text{max}}. \label{eq:pmus_bounds}
\end{align}
\end{subequations}

LS optimization problem. In other words, Another way to interpret the Equation~\ref{eq:cost_function} is that
the optimization solver is trying to obtain the closest possible estimative of $P_{\text{aw}}$
given the equation of motion, flow, volume and constraints on the unknown variables.
We can call the second term of the Equation~\ref{eq:cost_function} the estimative of airway pressure:





\begin{equation}
P_{\text{aw}}^{\text{est}}(t_k)  = \left(R\dot{V}(t_k) + E V(t_k) + \tilde{P}_{\text{mus}}(t_k)\right)
\label{eq:paw_est}
\end{equation}

~\cite{victorjr2023} improved the estimation and computational time by adding switching times:
explain Pawest -> equation


Another explored algorithm is the one proposed by~\cite{lv2025}. It is given as follows:

\section{Implementation environment}

Gurobi, YALMIP, MATLAB softwares, hardware used.

Code repository available in github/pmus-benchmarking.git (instructions of use, reproducibility).

\section{Proposed improvements}

This work proposes a improvement of using the information of more than one cycle,
as opposed to the current standard/understandings on non-invasive respiratory effort estimation.
For that purpose, excitation maneuvers will be applied in the experimental setting
with the objective to create variability and \hl{add more information} to the optimization problem.

\subsection{Maneuvers}

Description of all attempted maneuvers: P0.1, inspiratory pause, PS increase.

\subsection{Dual cycle estimation}

\section{Data acquisition}
\subsection{Benchtop experiment}
\begin{itemize}
    \item ASL 5000\textsuperscript{\texttrademark} Breathing Simulator is connected to a mechanical ventilator FlexiMag Max 700
(Magnamed\textsuperscript{\textregistered}, Sao Paulo, Brazil) configuration steps (pressure, resistance, compliance), ventilation modes, physical setup.
    \item Sampling frequency, acquisition duration, signal pre-processing.
\end{itemize}

\subsection{Patient data analysis (Restrospective (?))}
\begin{itemize}
    \item Dataset origin, ethical comittee number, anonymization.
    \item Patients description, inclusion/exclusion criteria, condition, modes, duration.
\end{itemize}

\section{Evaluation methods}

\subsection{Profile likelihood}

\subsection{Cost function heatmap}

\subsection{PMUS variability within cost valley}

\subsection{Global statistics}

PMUS Amplitudes and PTPmus Bland-Altman plot in experimental and patient data. \\
Performance comparison of the three algorithms Victor, Lv, Xia.
