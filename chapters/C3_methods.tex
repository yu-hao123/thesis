\chapter{MATERIALS AND METHODS}


\section{Algorithms formulation}

Assuming a first-order model for the respiratory system~\cite{bates2009}:
\begin{equation}
P_{\text{total}}(t) = P_{\text{aw}}(t) + P_{\text{mus}}(t) = E_{\text{rs}} \cdot V(t) + R_{\text{rs}} \cdot \dot{V}(t) + P_0
\label{eq:equation_motion}
\end{equation}

The explored model-based algorithms apply contrained optimization techniques in the respiratory equation of motion.
The problem of estimating $P_{\text{mus}}$, $R$, $E$ within a single cycle is indetermined ($N+2$ variables for $N$ equations as will be detailed below),
but the explored algorithms argument is that by introducing constraints on the $P_{\text{mus}}$ waveform format and the ranges of possible resistances and compliances, the problem becomes solvable.

~\cite{vicario2016} formulated the estimation problem as a constrained optimization problem with cost function:
\begin{equation}
J = \sum_{k=1}^{N}
\left(
P_{\text{aw}}(t_k) -\left(R\dot{V}(t_k) + E V(t_k) + \tilde{P}_{\text{mus}}(t_k)\right)
\right)^2
\label{eq:cost_function}
\end{equation}
to be minimized subject to the following constraints:
\begin{subequations}
\begin{align}
\tilde{P}_{\text{mus}}(t_{k+1}) - \tilde{P}_{\text{mus}}(t_k) &\le 0,
&& \text{for } k = 1,2,\ldots,m-1 \label{eq:constraint1} \\
\tilde{P}_{\text{mus}}(t_{k+1}) - \tilde{P}_{\text{mus}}(t_k) &\ge 0,
&& \text{for } k = m,m+1,\ldots,q-1 \label{eq:constraint2} \\
\tilde{P}_{\text{mus}}(t_{k+1}) - \tilde{P}_{\text{mus}}(t_k) &= 0,
&& \text{for } k = q,q+1,\ldots,N-1 \label{eq:constraint3}
\end{align}
\end{subequations}
where $t_k$ denotes the $k$th time sample, $N$ is the total number of time samples in the breath.
The motivation behind these constraints (Equations~\ref{eq:constraint1}–\ref{eq:constraint3}) is to define monotonicity of the muscle pressure format.
The parameters $t_m$ and $t_q$ define, respectively, the peak and end of the respiratory effort waveform.

The unknown variables over which $J$ is minimized are $R, E, \tilde{P}_{\text{mus}}(t_1), \ldots, \tilde{P}_{\text{mus}}(t_N)$.
Additional constraints limit $\tilde{P}_{\text{mus}}(t_k)$, $R$ and $E$ to positive values and within their physiological ranges:
\begin{subequations}
\begin{align}
R_{\text{min}} &\le R \le R_{\text{max}} \label{eq:R_bounds} \\
E_{\text{min}} &\le E \le E_{\text{max}} \label{eq:E_bounds} \\
\tilde{P}_{\text{min}} &\le \tilde{P}_{\text{mus}}(t_k) \le \tilde{P}_{\text{max}}. \label{eq:pmus_bounds}
\end{align}
\end{subequations}

LS optimization problem. In other words, Another way to interpret the Equation~\ref{eq:cost_function} is that
the optimization solver is trying to obtain the closest possible estimative of $P_{\text{aw}}$
given the equation of motion, flow, volume and constraints on the unknown variables.
We can call the second term of the Equation~\ref{eq:cost_function} the estimative of airway pressure:
\begin{equation}
P_{\text{aw}}^{\text{est}}(t_k)  = \left(R\dot{V}(t_k) + E V(t_k) + \tilde{P}_{\text{mus}}(t_k)\right)
\label{eq:paw_est}
\end{equation}

The algorithm proposed by~\cite{victorjr2023} reformulates the constrained optimization of~\cite{vicario2016}
as a mixed-integer quadratic programming (MIQP) problem. In this model, binary variables encode the switching
instants where the muscle pressure $P_{\text{mus}}(t)$ changes monotonicity, allowing
automatic identification of inspiratory, relaxation, and inactive phases within a single optimization step.
The method still minimizes the quadratic cost function of Equation~\ref{eq:cost_function},
subject to physiological bounds on $R$, $E$, and $P_{\text{mus}}(t)$.

The recent algorithm by~\cite{lv2025} applied this formulations in real-time clinical settings.
It maintains the first-order RC formulation but replaces the monotonic constraints with
a cubic polynomial model for $P_{\text{mus}}(t)$:
\begin{equation}
P_{\text{mus}}(t_k) = a_0 + a_1 t_k + a_2 t_k^2 + a_3 t_k^3
\label{eq:lv_cubic}
\end{equation}
subject to physiological shape constraints:
\begin{subequations}\label{eq:lv_constraints}
\begin{align}
P_{\text{mus}}(t_0) &= 0 \label{eq:lv_constraint1} \\
P_{\text{mus}}(t_m) &= 0 \label{eq:lv_constraint2} \\
P'_{\text{mus}}(t_0) &< 0 \label{eq:lv_constraint3} \\
P''_{\text{mus}}(t_0) &> 0 \label{eq:lv_constraint4} \\
P'''_{\text{mus}}(t_0) &< 0 \label{eq:lv_constraint5}
\end{align}
\end{subequations}

Additionally, the physiological bounds on resistance, elastance, and muscle pressure
follow the same form as Equations~\ref{eq:R_bounds}–\ref{eq:pmus_bounds}.

\section{Implementation environment}

All algorithms were implemented and executed in MATLAB\textsuperscript{\textregistered} R2024b,
using the YALMIP optimization framework and the commercial solver Gurobi Optimizer~(version~12.0.2).
Numerical routines, waveform processing, and visualization scripts were developed.

Computations were performed on a
Dell\textsuperscript{\textregistered} Inspiron laptop equipped with a
12th~Gen Intel\textsuperscript{\textregistered} Core\texttrademark~i5-1235U CPU (1.30~GHz, 10~cores) and 16~GB~RAM.

All code and data used in this study are publicly available
online\footnote{\url{https://github.com/yu-hao123/pmus-benchmarking}}
for reproducibility purposes. It includes the complete implementation scripts,
example datasets, and usage instructions.

\section{Proposed improvements}
This work proposes a improvement of using the information of more than one cycle,
as opposed to the current standard/understandings on non-invasive respiratory effort estimation.
For that purpose, excitation maneuvers will be applied in the experimental setting
with the objective to create variability and imporve the identifiability of the optimization problem.

\subsection{Maneuvers}

Description of all attempted maneuvers: P0.1, inspiratory pause, PS increase.

\subsection{Dual cycle estimation}

\section{Data acquisition}
\subsection{Benchtop experiment}
\begin{itemize}
    \item ASL 5000\textsuperscript{\texttrademark} Breathing Simulator is connected to a mechanical ventilator FlexiMag Max 700
(Magnamed\textsuperscript{\textregistered}, Sao Paulo, Brazil) configuration steps (pressure, resistance, compliance), ventilation modes, physical setup.
    \item Sampling frequency, acquisition duration, signal pre-processing.
\end{itemize}

\subsection{Patient data analysis (Restrospective (?))}
\begin{itemize}
    \item Dataset origin, ethical comittee number, anonymization.
    \item Patients description, inclusion/exclusion criteria, condition, modes, duration.
\end{itemize}

\section{Evaluation methods}

\subsection{Profile likelihood}

Following the definition in~\cite{raue2009,wieland2021}, the profile likelihood was
used to assess the practical identifiability of the estimated parameters.
For each parameter, the cost function was re-optimized over all remaining parameters
while varying the current parameter, and the resulting likelihood profile can be used
to identify flat regions and determine confidence interval boundaries (\Cref{fig:C3_profile_likelihood}).
\begin{figure}[ht]
    \caption{
        Illustrative example of likelihood contour plots and profile likelihood for an
        identifiable parameter and structurally and practically nonidentifiable parameters.
    }\label{fig:C3_profile_likelihood}
    \begin{center}
        \includegraphics[width=\textwidth]{chapters/figures/C3_profile_likelihood.jpg}
    \end{center}
    {\small Source:~\cite{wieland2021}}
\end{figure}{}

\subsection{Cost function heatmap}

A grid search was performed across predefined ranges of $R_{\text{rs}}$ and $E_{\text{rs}}$, applying fixed MIQP and cubic estimators.
The resulting cost landscape was visualized as a heatmap,
highlighting regions of minimum cost (valleys).

\subsection{$P_{\text{mus}}$ variability within cost valley}

Within the low-cost valley identified from the heatmap,
the variability of $P_{\text{mus}}$ amplitudes and $\text{PTP}_{\text{mus}}$ was evaluated.
For all points below a defined cost threshold, Bland-Altman plots were generated.

\subsection{Global statistics}

Bland-Altman plots of $P_{\text{mus}}$ amplitudes and $\text{PTP}_{\text{mus}}$ were applied to experimental and patient datasets.
Comparative performance of the three algorithms (MIQP, N-PMUS, dual) was assessed using bias and limits of agreement
across multiple breaths and subjects.