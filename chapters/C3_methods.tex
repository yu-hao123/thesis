\chapter{MATERIALS AND METHODS}


\section{Algorithms formulation}

Assuming a first-order model for the respiratory system~\cite{bates2009}:
\begin{equation}
P_{\text{total}}(t) = P_{\text{aw}}(t) + P_{\text{mus}}(t) = E_{\text{rs}} \cdot V(t) + R_{\text{rs}} \cdot \dot{V}(t) + P_0
\label{eq:equation_motion}
\end{equation}

The explored model-based algorithms apply contrained optimization techniques in the respiratory equation of motion.
The problem of estimating $P_{\text{mus}}$, $R$, $E$ within a single cycle is indetermined ($N+2$ variables for $N$ equations as will be detailed below),
but the explored algorithms argument is that by introducing constraints on the $P_{\text{mus}}$ waveform format and the ranges of possible resistances and compliances, the problem becomes solvable.

~\cite{vicario2016} formulated the estimation problem as a constrained least-squares optimization problem with cost function:
\begin{equation}
J = \sum_{k=1}^{N}
\left(
P_{\text{aw}}(t_k) -\left(R\dot{V}(t_k) + E V(t_k) + \tilde{P}_{\text{mus}}(t_k)\right)
\right)^2
\label{eq:cost_function}
\end{equation}
to be minimized subject to the following constraints:
\begin{subequations}
\begin{align}
\tilde{P}_{\text{mus}}(t_{k+1}) - \tilde{P}_{\text{mus}}(t_k) &\le 0,
&& \text{for } k = 1,2,\ldots,m-1 \label{eq:constraint1} \\
\tilde{P}_{\text{mus}}(t_{k+1}) - \tilde{P}_{\text{mus}}(t_k) &\ge 0,
&& \text{for } k = m,m+1,\ldots,q-1 \label{eq:constraint2} \\
\tilde{P}_{\text{mus}}(t_{k+1}) - \tilde{P}_{\text{mus}}(t_k) &= 0,
&& \text{for } k = q,q+1,\ldots,N-1 \label{eq:constraint3}
\end{align}
\end{subequations}
where $t_k$ denotes the $k$th time sample, $N$ is the total number of time samples in the breath
and $\tilde{P}_{\text{mus}} = P_{\text{mus}} - P_0$.
The motivation behind these constraints (Equations~\ref{eq:constraint1}–\ref{eq:constraint3}) is to define monotonicity of the muscle pressure format.
The parameters $t_m$ and $t_q$ define, respectively, the peak and end of the respiratory effort waveform.

The unknown variables over which $J$ is minimized are $R, E, \tilde{P}_{\text{mus}}(t_1), \ldots, \tilde{P}_{\text{mus}}(t_N)$.
Additional constraints limit $\tilde{P}_{\text{mus}}(t_k)$, $R$ and $E$ to positive values and within their physiological ranges:
\begin{subequations}
\begin{align}
R_{\text{min}} &\le R \le R_{\text{max}} \label{eq:R_bounds} \\
E_{\text{min}} &\le E \le E_{\text{max}} \label{eq:E_bounds} \\
\tilde{P}_{\text{min}} &\le \tilde{P}_{\text{mus}}(t_k) \le \tilde{P}_{\text{max}}. \label{eq:pmus_bounds}
\end{align}
\end{subequations}

Another way to interpret the Equation~\ref{eq:cost_function} is that
the optimization solver is trying to obtain the closest possible estimative of $P_{\text{aw}}$
given the equation of motion, flow, volume and constraints on the unknown variables.
We call the second term of the Equation~\ref{eq:cost_function} the estimative of airway pressure:
\begin{equation}
P_{\text{aw}}^{\text{est}}(t_k)  = \left(R\dot{V}(t_k) + E V(t_k) + \tilde{P}_{\text{mus}}(t_k)\right)
\label{eq:paw_est}
\end{equation}

The algorithm proposed by~\cite{victorjr2023} reformulates the constrained optimization of~\cite{vicario2016}
as a mixed-integer quadratic programming (MIQP) problem. In this model, binary variables encode the switching
instants where the muscle pressure $P_{\text{mus}}(t)$ changes monotonicity, allowing
automatic identification of inspiratory, relaxation, and inactive phases within a single optimization step.
The method still minimizes the quadratic cost function of Equation~\ref{eq:cost_function},
subject to physiological bounds on $R$, $E$, and $P_{\text{mus}}(t)$.

The recent algorithm by~\cite{lv2025} applied this formulations in real-time clinical settings.
It maintains the first-order $RC$ formulation but replaces the monotonic constraints with
a cubic polynomial model for $P_{\text{mus}}(t)$:
\begin{equation}
P_{\text{mus}}(t_k) = a_0 + a_1 t_k + a_2 t_k^2 + a_3 t_k^3
\label{eq:lv_cubic}
\end{equation}
subject to physiological shape constraints:
\begin{subequations}\label{eq:lv_constraints}
\begin{align}
P_{\text{mus}}(t_0) &= 0 \label{eq:lv_constraint1} \\
P_{\text{mus}}(t_m) &= 0 \label{eq:lv_constraint2} \\
P'_{\text{mus}}(t_0) &< 0 \label{eq:lv_constraint3} \\
P''_{\text{mus}}(t_0) &> 0 \label{eq:lv_constraint4} \\
P'''_{\text{mus}}(t_0) &< 0 \label{eq:lv_constraint5}
\end{align}
\end{subequations}

Additionally, the physiological bounds on resistance, elastance, and muscle pressure
follow the same form as Equations~\ref{eq:R_bounds}-\ref{eq:pmus_bounds}.

\section{Implementation environment}

All algorithms were implemented and executed in MATLAB\textsuperscript{\textregistered} R2024b,
using the YALMIP optimization framework and the commercial solver Gurobi Optimizer~(version~12.0.2).
Numerical routines, waveform processing, and visualization scripts were developed.

Computations were performed on a
Dell\textsuperscript{\textregistered} Inspiron laptop equipped with a
12th~Gen Intel\textsuperscript{\textregistered} Core\texttrademark~i5--1235U CPU (1.30~GHz, 10~cores) and 16~GB~RAM.

All code and data used in this study are publicly available
online\footnote{\url{https://github.com/yu-hao123/pmus-benchmarking}}
for reproducibility purposes. It includes the complete implementation scripts,
example datasets, and usage instructions.

\section{Modifications and proposed improvements}
This work proposes using the information and variability from multiple respiratory cycles simultaneously
within the optimization algorithm, in contrast to current standards in non-invasive respiratory effort estimation,
which rely on single cycle estimation.
To support this approach, excitation maneuvers were applied in the experimental setting
to introduce variability and improve the identifiability of the estimation problem.
\subsection{Maneuvers}\label{subsec:maneuvers}
\hl{Description of all attempted maneuvers, R, C shouldn't change in short periods of time}
(\Cref{fig:C3_maneuvers}): P0.1, inspiratory pause, PS increase.

\begin{figure}[H]
    \caption{
        Example of all three proposed maneuvers (P0.1, inspiratory pause and PS increase, in this order).
        Each maneuver is preceeded by a regular cycle.
    }\label{fig:C3_maneuvers}
    \begin{center}
        \includegraphics[width=\textwidth]{chapters/figures/C3_maneuvers.png}
    \end{center}
\end{figure}{}

\subsection{Optimization with fixed R, C values}\label{subsec:fixed_optimization}
The MIQP and N-PMUS methods were extended to allow optimization using fixed $R$ and $C$ values.
This approach is useful for evaluation methods that perform grid searches over the full range
of possible $R-C$ combinations. It also helps to visualize what $P_{\text{mus}}$ the estimator
would produce for a given pair of $R$ and $C$.

An important observation is that this method is not
equivalent to simply substituting fixed $R$, $C$, $P_{\text{aw}}$ and $\dot{V}$
directly into the equation of motion (\ref{eq:equation_motion}) to compute $P_{\text{mus}}$.
In the fixed-$RC$ optimization formulation, the solution space for $P_{\text{mus}}$ still satisfies
the imposed monotonicity constraints, whereas that would not be the case with direct substitution
in the equation of motion.

\subsection{Dual cycle estimation}
This work extended the original single-cycle MIQP formulation into a dual-cycle joint estimation approach.
Two consecutive respiratory cycles are concatenated into a single optimization problem that shares the
same resistance and elastance parameters. Each cycle maintains its own switching instants
(still implemented as binary variables with monotonicity constraints).
In this way, information from both cycles are incorporated into the optimization problem.

The proposed methods evaluates which of the maneuvers described in \Cref{subsec:maneuvers}
are more effective for improving the identifiability
of the estimation problem.
Dual-cycle estimation is applied to pairs of regular
and subsequent maneuvered cycles.
For example, in \Cref{fig:C3_maneuvers} there are three of these pairs.


\section{Data acquisition}
\subsection{Benchtop experiment}
\begin{itemize}
    \item ASL 5000\textsuperscript{\texttrademark} Breathing Simulator is connected to a mechanical ventilator FlexiMag Max 700
(Magnamed\textsuperscript{\textregistered}, Sao Paulo, Brazil) configuration steps (pressure, resistance, compliance), ventilation modes, physical setup.
    \item ASL 5000\textsuperscript{\texttrademark} was also connected to a NKV-550
    (NIHON KOHDEN\textsuperscript{\textregistered}, Tokyo, Japan) mechanical ventilator.
    \item Sampling frequency, acquisition duration, signal pre-processing.
\end{itemize}

\hl{Discuss acquiring data in multiple modes of ventilation (VCV, PCV, PSV).}

\subsection{Patient data analysis}

This work analyzes a retrospective clinical dataset~\cite{plens2024}
 of esophageal pressure measurements in adults in pressure support ventilation.
\begin{itemize}
    \item Dataset origin, ethical comittee number, anonymization.
    \item Patients description, inclusion/exclusion criteria, condition, modes, duration.
\end{itemize}

\section{Evaluation metrics}

The following evaluation metrics are used to compare the different estimation algorithms.
Some of these metrics are standard in the respiratory effort estimation literature,
while others have not been previously applied to effort-estimation algorithms,
but provide additional perspectives with respect to on the algorithms' ability
to resolve the estimation problem's indetermined nature.

\subsection{Profile likelihood}

Following the definition in~\cite{raue2009,wieland2021}, the profile likelihood is
used to assess the practical identifiability of the estimated parameters.
For each parameter, the cost function is re-optimized over all remaining parameters
while varying the current parameter, and the resulting likelihood profile can be used
to observe flat regions and determine confidence interval boundaries (\Cref{fig:C3_profile_likelihood}).
\begin{figure}[ht]
    \caption{
        Illustrative example of likelihood contour plots and profile likelihood for an
        identifiable parameter and structurally and practically nonidentifiable parameters.
    }\label{fig:C3_profile_likelihood}
    \begin{center}
        \includegraphics[width=\textwidth]{chapters/figures/C3_profile_likelihood.jpg}
    \end{center}
    {\small Source:~\cite{wieland2021}}
\end{figure}{}

\subsection{Cost function residual surface}

A grid search can be performed across predefined ranges of $R_{\text{rs}}$ and $E_{\text{rs}}$,
applying the fixed MIQP and fixed cubic estimators, as described in \Cref{subsec:fixed_optimization}.
The resulting cost surface is visualized as a heatmap,
highlighting regions of minimum cost (valleys).

\subsection{$P_{\text{mus}}$ variability within residual cost surface valley}

Within the low-cost valley identified from the residual cost surface,
the variability of $P_{\text{mus}}$ amplitudes and $\text{PTP}_{\text{mus}}$ is evaluated.
All points below a defined cost threshold are included in a Bland-Altman plot.

\subsection{Statistical analysis}

Bland-Altman plots of $P_{\text{mus}}$ amplitudes and $\text{PTP}_{\text{mus}}$ are applied to experimental and patient datasets.
Performance comparisons of the three algorithms (MIQP, N-PMUS-cubic, dual) are done using bias and limits of agreement
across multiple breaths and subjects.