\chapter{HYPOTHESIS AND OBJECTIVES}

We hypothesize that:

\begin{itemize}
    \item Overly restrictive formulations in constrained optimization may cause
    overfitting and generate respiratory effort estimatives that deviate from the true physiological condition.
    \item The monotonic formulation proposed by~\cite{vicario2016} applied in isolated
    cycles is insufficient to guarantee a unique and physiologically consistent estimative.
    \item Incorporating multiple respiratory cycles or controlled maneuvers into the estimation algorithm can reduce ambiguity.
    \item Identifiability analysis provides a complementary approach to evaluate the estimation algorithms,
    offering perspectives that cannot be obtained solely from statistical validation on experimental and clinical data.
\end{itemize}

The main objectives of this work are:

\begin{itemize}
    \item Reproduce and compare existing model-based constrained optimization formulations for
    noninvasive estimation of respiratory effort, comparing them using both experimental and clinical data through statistical analysis.
    \item Investigate ambiguities, robustness and identifiability, evaluating how model assumptions
    and data characteristics influence the uniqueness and physiological plausibility of the estimated effort.
    \item Propose methodological improvements to improve reliability and reduce ambiguity of noninvasive effort monitoring.
\end{itemize}