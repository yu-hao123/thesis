\chapter{INTRODUCTION}

Respiratory effort monitoring is recognized as an essential tool for individualizing
mechanical ventilator settings in lung and diaphragm protective ventilation strategies
~\cite{vanoosten2024, cornejo2024}.
Monitoring is crucial because both excessive and low inspiratory efforts are associated with
multiple types of lung and diaphragm injuries including patient self-inflicted lung injury (P-SILI)
and myotrauma~\cite{tonelli2025,yoshida2020b}, and mismatches/asynchronies can induce traumas to
both organs. Thus, the primary aim of such monitoring is to maintain in the patient a
moderate amount of respiratory drive to prevent mechanical ventilation-induced complications and
help in weaning~\cite{jonkman2020}.

Explore monitoring (esophageal~\cite{akoumianaki2014,mauri2016}, maneuver surrogates PO.1 and Pooc) and estimation methods.

There are multiple estimators that apply some form of
constrained optimization approach~\cite{vicario2016,victorjr2023,lv2025}.

Recent developments published clinical validations of such methods~\cite{lv2025}.

Analyze weaknesses and possible knowledge gaps in noninvasive estimation.

Show identifiability methods~\cite{raue2009,wieland2021}.

Thus the purpose of this work is to reproduce literature noninvasive muscle estimation,
explore the robustness and identifiability of such formulations and propose improvements.

