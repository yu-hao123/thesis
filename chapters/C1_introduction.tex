\chapter{INTRODUCTION}

Respiratory effort monitoring is recognized as an essential tool for individualizing
mechanical ventilator settings in lung and diaphragm protective ventilation strategies
~\cite{vanoosten2024, cornejo2024}.
Monitoring is crucial because both excessive and low inspiratory efforts are associated with
multiple types of lung and diaphragm injuries including patient self-inflicted lung injury (P-SILI)
and myotrauma~\cite{tonelli2025,yoshida2020b}, and mismatches/asynchronies can induce traumas to
both organs. Thus, the primary aim of such monitoring is to maintain in the patient a
moderate amount of respiratory drive to prevent mechanical ventilation-induced complications and
help in weaning~\cite{jonkman2020}.

Explore monitoring (esophageal~\cite{akoumianaki2014,mauri2016}, maneuver surrogates PO.1 and Pooc) and estimation methods.

There are multiple estimators that apply some form of
constrained optimization approach~\cite{vicario2016,victorjr2023,lv2025}.

Recent developments published clinical validations of such methods~\cite{lv2025}.

Analyze weaknesses and possible knowledge gaps in noninvasive estimation.

Show identifiability methods~\cite{raue2009,wieland2021}.

Thus the purpose of this work is to reproduce literature noninvasive muscle estimation,
explore the robustness and identifiability of such formulations and propose improvements.

\begin{figure}[ht]
	\caption{
        Space delimitation
    }\label{fig:circulo}
	\begin{center}
		\setlength{\unitlength}{9cm}
		\begin{picture}(1,1)
		\put(0,0){\line(0,1){1}}
		\put(0,0){\line(1,0){1}}
		\put(0,0){\line(1,1){1}}
		\put(0,0){\line(1,2){.5}}
		\put(0,0){\line(1,3){.3333}}
		\put(0,0){\line(1,4){.25}}
		\put(0,0){\line(1,5){.2}}
		\put(0,0){\line(1,6){.1667}}
		\put(0,0){\line(2,1){1}}
		\put(0,0){\line(2,3){.6667}}
		\put(0,0){\line(2,5){.4}}
		\put(0,0){\line(3,1){1}}
		\put(0,0){\line(3,2){1}}
		\put(0,0){\line(3,4){.75}}
		\put(0,0){\line(3,5){.6}}
		\put(0,0){\line(4,1){1}}
		\put(0,0){\line(4,3){1}}
		\put(0,0){\line(4,5){.8}}
		\put(0,0){\line(5,1){1}}
		\put(0,0){\line(5,2){1}}
		\put(0,0){\line(5,3){1}}
		\put(0,0){\line(5,4){1}}
		\put(0,0){\line(5,6){.8333}}
		\put(0,0){\line(6,1){1}}
		\put(0,0){\line(6,5){1}}
		\end{picture}
	\end{center}
    \vspace{2mm}
    {\small Source: Solid line - simulated data; dashed line - experimental data.}
\end{figure}
