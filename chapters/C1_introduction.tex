\chapter{INTRODUCTION}

Respiratory effort monitoring is recognized as an essential tool for individualizing
mechanical ventilator settings in lung and diaphragm protective ventilation strategies
~\cite{vanoosten2024, cornejo2024}.
Monitoring is crucial because both excessive and low inspiratory efforts are associated with
multiple types of lung and diaphragm injuries including patient self-inflicted lung injury (P-SILI)
and myotrauma~\cite{tonelli2025,yoshida2020b}, and mismatches/asynchronies can induce traumas to
both organs. Thus, the primary aim of such monitoring is to maintain in the patient a
moderate amount of respiratory drive to prevent mechanical ventilation-induced complications and
help in weaning~\cite{jonkman2020}.

In clinical practice, respiratory effort can be directly monitored using
esophageal pressure measurements~\cite{akoumianaki2014,mauri2016},
or indirectly through surrogate maneuvers such as the 100-milliseconds airway occlusion pressure (P0.1)
and full-breath occluded airway pressure drop (Pocc)~\cite{bertoni2019,telias2020}.
While esophageal manometry provides the most accurate measure of muscle effort, its invasiveness,
calibration complexities and limited clinical use motivate the development of noninvasive alternatives.
In contrast, P0.1 and Pocc reflect the patient's neural drive and the presence of
excessive or insufficient effort, rather than providing a direct quantitative estimate
of muscle pressure~\cite{vanoosten2024}.

There is a growing amount of research focusing on the noninvasive estimation of respiratory effort
using model-based approaches. Several studies apply constrained optimization formulations~\cite{vicario2016,victorjr2023,lv2025}.
to obtain muscle pressure and respiratory mechanics parameters (resistance and compliance) from pressure, flow and volume signals.
Recent developments even published clinical validations of such methods~\cite{lv2025}.

Despite these advances, challenges remain regarding the robustness and identifiability
of noninvasive respiratory muscle pressure estimation. Identifiability analysis, which determines
whether model parameters can be uniquely determined from available data, is essential for ensuring
meaningful physiological interpretation~\cite{raue2009,wieland2021}.
Therefore, this work aims to reproduce existing noninvasive effort estimation algorithms,
compare and evaluate them with practical identifiability techniques and statistical analysis
in experimetnal data. This work also propose and assess methodological improvements to these algorithms.